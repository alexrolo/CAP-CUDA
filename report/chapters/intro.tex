\pagestyle{fancy}
\fancyhead[LO]{\autorR}
\fancyhead[LE]{\autorA}
\fancyhead[RE,RO]{\textit{\rightmark}}
\fancyfoot[L]{\asignaturaAbbr}
\fancyfoot[R]{\fecha}

\section{Introducción} \label{sec:1}
En esta sesión de prácticas de laboratorio se aborda la programación en C/CUDA. Para ello, se implementarán y analizarán 
dos problemas:
\vspace{0.1cm}
\begin{itemize}
    \item Producto matricial.
    \item Resolución de sistemas de ecuaciones utilizando el método de \textit{Gauss-Jordan}.
\end{itemize}
\vspace{0.1cm}
El objetivo principal de esta sesión es la implementación de ambos problemas en C/CUDA y su posterior análisis de rendimiento 
respecto a la implementación en CPU.

\subsection{Desarrollo}
Para llevar a cabo el desarrollo de esta práctica, se han seguido las indicaciones recogidas en el guion de la sesión correspondiente.
Cada uno de los dos alumnos involucrados se ha centrado en la resolución de uno de los problemas, 
siendo el producto matricial el problema asignado al alumno \autorR\ y la resolución de sistemas de ecuaciones el problema
asignado al alumno \autorA.

Todo el código fuente se encuentra disponible públicamente en el siguiente 
\href{https://github.com/alexrolo/CAP-CUDA}{repositorio de GitHub}, así como en el archivo \textit{zip} asociado a esta entrega.

\subsection{\textit{Benchmarking}}
Con fines de realizar análisis de rendimiento, se han implementado \textit{benchmarks} para simplificar la ejecución y la toma 
de tiempos con diferentes configuraciones y datos de entrada en ambos problemas.

\subsubsection{Producto matricial}
Respecto al producto matricial, se ha implementado un programa en C/CUDA (\textit{cuda\_mul\_benchmark.cu}) para cumplir con el 
propósito anterior. Este programa permite la ejecución del producto de matrices de tamaño $N \times N$ con diferentes configuraciones
de \textit{threads per block}. Además, se define un número de iteraciones para cada operación a realizar, de forma que se pueda 
obtener un tiempo medio de ejecución de todas ellas, minimizando así el impacto de la variabilidad en el rendimiento.
De forma sencilla, modificando una simple variable (\textit{ONLY\_KERNEL\_TIME}), se puede elegir si se desea medir el tiempo 
de ejecución del \textit{kernel} únicamente o si se desea incluir el tiempo de reserva de memoria y el tiempo de copia de resultados.
El programa devuelve un \textit{log} en formato CSV con los resultados obtenidos.

\subsubsection{Gauss Jordan}
Para el problema de resolución de sistemas de ecuaciones lineales mediante el método de Gauss Jordan,
se han implementado los programas \textit{equations.c} para la versión en C y
\textit{equations\_cuda.cu} para la versión en CUDA,
cumpliendo con los requisitos de la práctica.
Para la obtención de métricas, se han utilizado los programas
\textit{comparer.cpp} y \textit{comparer\_cuda.cu} respectivamente.
Ambos programas son similares, iterando por una serie de tamaños de matriz y
repitiendo los cálculos un total de 32 veces para obtener
una media de tiempos de ejecución.
Nótese que para ambos programas se ha utilizado la librería estándar de C++
para simplificar el trabajo con \texttt{std::vector}.

