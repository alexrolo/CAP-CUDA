\pagestyle{fancy}
\fancyhead[LO]{\autorR}
\fancyhead[LE]{\autorA}
\fancyhead[RE,RO]{\textit{\rightmark}}
\fancyfoot[L]{\asignaturaAbbr}
\fancyfoot[R]{\fecha}

\section{Introducción} \label{sec:1}
En esta sesión de prácticas de laboratorio se aborda la programación en C/CUDA. Para ello, se implementarán y analizarán 
dos problemas:
\vspace{0.1cm}
\begin{itemize}
    \item Producto matricial.
    \item Resolución de sistemas de ecuaciones utilizando el método de \textit{Gauss-Jordan}.
\end{itemize}
\vspace{0.1cm}
El objetivo principal de esta sesión es la implementación de ambos problemas en C/CUDA y su posterior análisis de rendimiento 
respecto a la implementación en CPU.

\subsection{Desarrollo}
Para llevar a cabo el desarrollo de esta práctica, se han seguido las indicaciones recogidas en el guion de la sesión correspondiente.
Cada uno de los dos alumnos involucrados se ha centrado en la resolución de uno de los problemas, 
siendo el producto matricial el problema asignado al alumno \autorR\ y la resolución de sistemas de ecuaciones el problema
asignado al alumno \autorA.

Todo el código fuente se encuentra disponible públicamente en el siguiente 
\href{https://github.com/alexrolo/CAP-CUDA}{repositorio de GitHub}, así como en el archivo \textit{zip} asociado a esta entrega.

\subsection{\textit{Benchmarking}}
